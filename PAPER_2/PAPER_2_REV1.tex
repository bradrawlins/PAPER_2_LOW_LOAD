\documentclass[review]{elsarticle}
\usepackage[utf8]{inputenc} 
\usepackage{lineno,hyperref}
\usepackage{tabu}
\usepackage{tabularx}
\usepackage{subfigure}
\modulolinenumbers[5]

\journal{Journal of \LaTeX\ Templates}

%%%%%%%%%%%%%%%%%%%%%%%
%% Elsevier bibliography styles
%%%%%%%%%%%%%%%%%%%%%%%
%% To change the style, put a % in front of the second line of the current style and
%% remove the % from the second line of the style you would like to use.
%%%%%%%%%%%%%%%%%%%%%%%

%% Numbered
%\bibliographystyle{model1-num-names}

%% Numbered without titles
%\bibliographystyle{model1a-num-names}

%% Harvard
%\bibliographystyle{model2-names.bst}\biboptions{authoryear}

%% Vancouver numbered
%\usepackage{numcompress}\bibliographystyle{model3-num-names}

%% Vancouver name/year
%\usepackage{numcompress}\bibliographystyle{model4-names}\biboptions{authoryear}

%% APA style
%\bibliographystyle{model5-names}\biboptions{authoryear}

%% AMA style
%\usepackage{numcompress}\bibliographystyle{model6-num-names}

%% `Elsevier LaTeX' style
\bibliographystyle{elsarticle-num}
%%%%%%%%%%%%%%%%%%%%%%%

\begin{document}

\begin{frontmatter}

\title{Low load operating protocol investigation of a 620MWe power boiler using a fast Eulerian-Eulerian CFD model}

%% Group authors per affiliation:
\author{B.T. Rawlins}
\author{R. Laubscher\corref{mycorrespondingauthor}}
\cortext[mycorrespondingauthor]{Corresponding author}
\ead{ryno.laubscher@uct.ac.za}
\author{P. Rousseau}
\address{Department of Mechanical Engineering, Applied Thermal-Fluid Process Modeling Research Unit, University of Cape Town, Library Rd, Rondebosch, Cape Town, 7701, South Africa}


\begin{abstract}

This template helps you to create a properly formatted \LaTeX\ manuscript.
Low load operation of utility boiler

\end{abstract}

\begin{keyword}
CFD\sep Eulerian-Eulerian\sep Boiler \sep Low-load operation
\end{keyword}

\end{frontmatter}

\linenumbers

\begin{center}
\begin{tabular}{|p{0.1\textwidth}p{0.25\textwidth}p{0.05\textwidth}p{0.1\textwidth}p{0.25\textwidth}p{0.05\textwidth}|} 
 \hline
\multicolumn{3}{|c}{\textbf{Nomenclature}} & &  &\\
& & & & & \\
& &  & \multicolumn{3}{l|}{\textit{Greek letters}}\\
\textit{Symbol} & \textit{Quantity} & \textit{Unit} & $\alpha_p$ & Particle absorption coefficient &$m^{-1}$\\
A &  Area & $m^2$ & g & Gravity  & $m/s^2$\\
 \hline
\end{tabular}
\end{center}

\section{Introduction}

If the document class \emph{elsarticle} is not available on your computer, you can download and install the system package \emph{texlive-publishers} (Linux) or install the \LaTeX\ package \emph{elsarticle} using the package manager of your \TeX\ installation, which is typically \TeX\ Live or Mik\TeX.

\paragraph{Usage} Once the package is properly installed, you can use the document class \emph{elsarticle} to create a manuscript. Please make sure that your manuscript follows the guidelines in the Guide for Authors of the relevant journal. It is not necessary to typeset your manuscript in exactly the same way as an article, unless you are submitting to a camera-ready copy (CRC) journal.

\paragraph{Functionality} The Elsevier article class is based on the standard article class and supports almost all of the functionality of that class. In addition, it features commands and options to format the
\begin{itemize}
\item document style
\item baselineskip
\item front matter
\item keywords and MSC codes
\item theorems, definitions and proofs
\item lables of enumerations
\item citation style and labeling.
\end{itemize}

\section{Mathematical model}

The author names and affiliations could be formatted in two ways:
\begin{enumerate}[(1)]
\item Group the authors per affiliation.
\item Use footnotes to indicate the affiliations.
\end{enumerate}
See the front matter of this document for examples. You are recommended to conform your choice to the journal you are submitting to.

\section{Numerical setup}

\begin{table}[h!]
\centering
\caption{Utility boiler fuel characteristics}
\vspace{5mm}
\label{fuel}
{\tabulinesep=1.2mm
\begin{tabularx}{\textwidth}{p{0.45\textwidth} p{0.3\textwidth} l}
\hline
\textbf{Fuel constituent} & \textbf{Fraction} & \textbf{Unit}\\
\hline
\textit{Ultimate analysis - (DAF)} & \textit{-} & \textit{-}\\
Carbon & $0.7753$ & $kg/kg_{fuel}$\\
Hydrogen & $0.0415$ & $kg/kg_{fuel}$\\
Nitrogen & $0.0181$ & $kg/kg_{fuel}$\\
Oxygen & $0.1474$ & $kg/kg_{fuel}$\\
Sulphur & $0.0175$ & $kg/kg_{fuel}$\\
\textit{Proximate analysis - (AR)} & \textit{-} & \textit{-}\\
Fixed carbon & $0.340$ & $kg/kg_{fuel}$\\
Volatile matter & $0.196$ & $kg/kg_{fuel}$\\
Ash & $0.4090$ & $kg/kg_{fuel}$\\
Moisture & $0.0550$ & $kg/kg_{fuel}$\\
\hline
\textbf{Energy content - (DAF)} & \textbf{Value} &\\
\hline
Higher heating value & $15070$ & $kJ/kg_{fuel}$\\
\hline
\end{tabularx}}
\end{table}



Validation separately mention the rates/loadings - give results for 40\% case inputs
Low Ultra low load inputs


\subsection{Model validation}
The validation of the proposed model was conducted for three steady state MCR loads of 100\%, 80\% and 60\%. The  
\\
Figures of the Histrograms and CO graphs as in paper
Data plots and overall performance of the model
Inputs for the various loads table

\begin{figure}\label{fig_heat_valid}
	\subfigure[]{\includegraphics[width=0.32\textwidth]{100_VALID_BAR}}
	\subfigure[]{\includegraphics[width=0.32\textwidth]{80_VALID_BAR}}
	\subfigure[]{\includegraphics[width=0.32\textwidth]{60_VALID_BAR}}
\caption{Validation of the experimental and models heat load to the furnace, PSH and HTSH for steady state loads of a) 100\% MCR, b) 80\% MCR and c) 60\% MCR}
\end{figure}

\begin{figure}\label{fig_comb_valid}
\centering
\subfigure[]{\includegraphics[width = 0.45\textwidth]{COPPM_VALID}}
\subfigure[]{\includegraphics[width = 0.45\textwidth]{XO2_VALID}}
\caption{Experimentally calculated $CO$ and $O_2$ concentration predictions at a height of 37.5 (m)}
\end{figure}


\begin{figure}
\centering
\includegraphics[scale = 0.5]{TEMP_KEY}\\
\subfigure[]{\includegraphics[width=0.32\textwidth]{BOT_TEMP}}
\subfigure[]{\includegraphics[width=0.32\textwidth]{MID_TEMP}}
\subfigure[]{\includegraphics[width=0.32\textwidth]{FBRM_TEMP}}\\
\subfigure[]{\includegraphics[width=0.32\textwidth]{BOT05_TEMP}}
\subfigure[]{\includegraphics[width=0.32\textwidth]{MID05_TEMP}}
\subfigure[]{\includegraphics[width=0.32\textwidth]{FBRM05_TEMP}}
\caption{Hi}
\end{figure}

ghff
\begin{figure}[h!]
\centering
\includegraphics[scale = 0.5]{VEL_KEY}\\
\subfigure[]{\includegraphics[width=0.32\textwidth]{BOT_VEL}}
\subfigure[]{\includegraphics[width=0.32\textwidth]{MID_VEL}}
\subfigure[]{\includegraphics[width=0.32\textwidth]{FBRM_VEL}}\\
\subfigure[]{\includegraphics[width=0.32\textwidth]{BOT05_VEL}}
\subfigure[]{\includegraphics[width=0.32\textwidth]{MID05_VEL}}
\subfigure[]{\includegraphics[width=0.32\textwidth]{FBRM05_VEL}}
\caption{bye}
\end{figure}

\begin{figure}[h!]
\centering
\includegraphics[scale = 0.5]{HEATFLUX_KEY}\\
\subfigure[]{\includegraphics[width=0.32\textwidth]{BOT_HEATFLUX}}
\subfigure[]{\includegraphics[width=0.32\textwidth]{MID_HEATFLUX}}
\subfigure[]{\includegraphics[width=0.32\textwidth]{FBRM_HEATFLUX}}\\
\subfigure[]{\includegraphics[width=0.32\textwidth]{BOT05_HEATFLUX}}
\subfigure[]{\includegraphics[width=0.32\textwidth]{MID05_HEATFLUX}}
\subfigure[]{\includegraphics[width=0.32\textwidth]{FBRM05_HEATFLUX}}
\caption{bye}
\end{figure}

\section{Results \& discussion}

\section{Conclusions}

\section{Bibliography styles}

There are various bibliography styles available. You can select the style of your choice in the preamble of this document. These styles are Elsevier styles based on standard styles like Harvard and Vancouver. Please use Bib\TeX\ to generate your bibliography and include DOIs whenever available.

Here are two sample references: \cite{Feynman1963118,Dirac1953888}.

\section*{References}

\bibliography{mybibfile}

\end{document}